% !TEX root = Fragebogen_1.tex

\section{\uppercase{Studienwahl}}
\vspace{.25cm}

\begin{choicequestion}[1]{Welchen Bachelor of Education studieren Sie?}
	\choiceitem{Primarstufe}
	\choiceitem{Sekundarstufe}
\end{choicequestion}

\separate

\begin{choicequestion}[1]{Studieren Sie mit der Profilierung Europalehramt?}
	\choiceitem{ja}
	\choiceitem{nein}
\end{choicequestion}

\separate

\begin{choicequestion}[1]{Planen Sie als Lehrer/in tätig zu sein (den Master of Education anzuschließen)?}
	\choiceitem{ja}
	\choiceitem{nein}
\end{choicequestion}

\separate

%todo Klären, wie mit dem Wahlbereich Primarstufe MATHE/DEUTSCH umzugehen ist.
\begin{choicequestion}[1]{Welche Fächerkombination haben Sie gewählt?}
	\choiceitemtext{1cm}{1}{1. Fach:}
	\choiceitemtext{1cm}{1}{2. Fach:}
\end{choicequestion}

\separate

\begin{choicequestion}[1]{Ist das die von Ihnen am meisten gewünschte Fächerkombination?}
	\choiceitem{ja}
	\choiceitem{nein}
\end{choicequestion}

\separate

\begin{choicequestion}[1]{Handelt es sich um Ihr erstes Studiensemester an der PH Karlsruhe?}
	\choiceitem{ja}
	\choiceitem{nein}
\end{choicequestion}

\separate

\begin{choicequestion}[1]{Haben Sie an der Orientierungsphase (O-Phase) der PH/StuVe teilgenommen?}
	\choiceitem{vollständig (Montag bis Samstag)}% TAGE KLÄREN
	\choiceitem{teilweise (mindestens ein Tag)}
	\choiceitem{gar nicht}
\end{choicequestion}
