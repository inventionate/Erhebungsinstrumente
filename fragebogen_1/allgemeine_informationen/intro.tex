% !TEX root = ../Fragebogen_1.tex

% Metadaten für den Auswertungsbericht
\addinfo{Datum}{22.07.2015}
\addinfo{Befragungssituation}{Zweiter Pretest (Intensiver Pretest mit fünf Studierenden)}

% Introtext
\begin{info}
\vspace{.5cm}

Sehr geehrte Studentinnen und Studenten,

\vspace{.5em}

der folgende Fragebogen bezieht sich auf Ihren »Zeit-Raum Studium« an der Pädagogischen Hochschule Karlsruhe. Die Erhebung und Auswertung findet im Rahmen meiner Doktorarbeit an der Pädagogischen Hochschule Karlsruhe in der Allgemeinen und Historischen Erziehungswissenschaft statt. Geplant ist, Sie in den ersten drei Semestern Ihres Bachelorstudiums forschend zu begleiten. Dabei interessieren mich insbesondere Ihre zeitlichen und räumlichen Erfahrungen. Durch diese erhoffe ich mir den Zeit-Raum Studium besser verstehen zu lernen. Dieses tiefere Verständnis soll letztlich auch Ihnen helfen, indem es ganz konkret auf die Verbesserung der Studienbedingungen vor Ort abzielt.

\vspace{.5em}

Der Fragebogen gliedert sich in \textbf{drei Teile}. Zunächst wird nach Ihrem \textbf{biografischen Hintergrund} gefragt, danach stehen Fragen zu Ihren Erfahrungen in der \textbf{Übergangszeit zwischen Schule und Hochschule} im Mittelpunkt. Den Abschluss bilden Fragen zu Ihren \textbf{alltäglichen Studienaktivitäten}.

\vspace{.5em}

Ich bitte Sie, sich Zeit und Raum zu nehmen und den Fragebogen ehrlich und vollständig zu bearbeiten. Die Auswertung erfolgt anonym. Da die Studie längsschnittlich (über drei Semester) angelegt ist und zudem einige Interviews durchgeführt werden sollen, wird nach Ihrer E-Mail-Adresse  gefragt. Diese personenbezogene Angabe wird getrennt von Ihren übrigen Antworten aufbewahrt und wird lediglich zur Kontaktaufnahme für eine Interviewanfrage herangezogen.

\vspace{.5em}

Vielen Dank im Voraus für Ihre Bereitschaft, Hilfe und Unterstützung!

\vspace{-1cm}

\includegraphics[width=0.25\textwidth]{unterschrift.png}

\vspace{-1.5cm}
\end{info}

\vspace{.75cm}

% Bearbeitungshinweise
\begin{info}
\vspace{.5em}

\textbf{\Large Bearbeitungshinweise}

Der Fragebogen wird maschinell erfasst. Um dabei optimale Ergebnisse zu erzielen, verwenden Sie bitte einen \textbf{Kugelschreiber} oder einen \textbf{nicht zu dicken Filzstift} (es muss ein Kreuz erkennbar sein) zum ausfüllen.

\vspace{.5em}

Wichtig ist, dass Sie ein \textbf{kräftiges (möglichst dunkles)} und \textbf{deutlich erkennbares Kreuz} setzen. Bitte verwenden Sie daher keine blassen Stifte (\textbf{keine Bleistifte}).

\vspace{.5em}

Beispiel für ein gut erfassbares Kreuz: {\LARGE \checkedbox{}}

\vspace{.5em}

Sie können einmalig Ihre Markierung korrigieren, indem Sie das Feld komplett ausfüllen.

\vspace{.5em}

Beispiel für eine Korrektur (Kreuz wird nicht erfasst): {\LARGE \correctedbox{}}

\vspace{.5em}

Sollten Sie Ihre Angaben mehrfach korrigieren wollen, notieren Sie ihre Angaben bitte handschriftlich.

\vspace{.5em}

\textbf{Bitte beachten Sie, dass der Fragebogen doppelseitig bedruckt ist. Bitte achten Sie auch darauf, die QR-Codes und die rechtwinkligen Seitenbegrenzungen nicht zu beschädigen.}

\end{info}
