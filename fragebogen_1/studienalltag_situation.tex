% !TEX root = Fragebogen_1.tex

% Persönliche Situation
\section{Persönliche Situation}
\vspace{.25cm}

\begin{choicequestion}[1]{Wo wohnen Sie \underline{während der Vorlesungszeit?}}
	\choiceitem{bei meinen Eltern}
	\choiceitem{in einer WG}
	\choiceitem{in einem Wohnheim}
	\choiceitem{allein in einer eigenen Wohnung}
	\choiceitem{mit meinem Partner/meiner Partnerin}
	\choiceitem{woanders}
\end{choicequestion}

\separate

\begin{choicequestion}[6]{Wie viele Kilometer sind es ungefähr von Ihrem Wohnort \underline{während der Vorlesungszeit} bis zur Hochschule?}
	\choiceitemtext{1cm}{1}{Kilometer:}
\end{choicequestion}

\separate

\begin{choicequestion}[6]{Wie lange brauchen Sie ungefähr von Ihrem Wohnort \underline{während der Vorlesungszeit} bis zur Hochschule?}
	\choiceitemtext{1cm}{1}{Minuten:~~~}
\end{choicequestion}

\separate

\begin{choicequestion}[1]{Wo wohnen Sie aller Voraussicht nach \underline{während der »Semesterferien«?}}
	\choiceitem{bei meinen Eltern}
	\choiceitem{in einer WG}
	\choiceitem{in einem Wohnheim}
	\choiceitem{allein in einer eigenen Wohnung}
	\choiceitem{mit meinem Partner/meiner Partnerin}
	\choiceitem{woanders}
	%\choiceitemtext{1cm}{1}{anderer Ort:}
\end{choicequestion}

\separate

\begin{choicequestion}[6]{Wie viele Kilometer sind es ungefähr von Ihrem Wohnort \underline{während der »Semesterferien«} bis zur Hochschule?}
	\choiceitemtext{1cm}{1}{Kilometer:}
\end{choicequestion}

\separate

\begin{choicequestion}[6]{Wie lange brauchen Sie ungefähr von Ihrem Wohnort \underline{während der »Semesterferien«} bis zur Hochschule?}
	\choiceitemtext{1cm}{1}{Minuten:~~~}
\end{choicequestion}

\separate

\begin{choicegroup}{Wie finanzieren Sie derzeit Ihr Studium?}
	\groupaddchoice{nein, dadurch nicht}
	\groupaddchoice{ja, teilweise}
	\groupaddchoice{ja, hauptsächlich}

	\choiceline{durch Unterstützung der Eltern}
	\choiceline{durch Einkommen des/ der (Ehe-)Partners/-in} %/der (Ehe-)Partnerin}
	\choiceline{durch BAföG}
	\choiceline{durch Studienkredit/ Bildungskredit}
	\choiceline{durch Stipendien (Förderwerke, Stiftungen,…)}
%	\choiceline{durch die Arbeit als studentische Hilfskraft}
	\choiceline{durch eigene Arbeit während der Vorlesungszeit}
	\choiceline{durch eigene Arbeit während der Semesterferien}
	\choiceline{durch anderes}
\end{choicegroup}

\separate

\textbox*{1cm}{Falls Sie einer bezahlten Arbeit nachgehen, um welche handelt es sich?}

\separate

\singlemarkLabelsSixLine{Wie kommen Sie insgesamt mit dem Geld zurecht, das Ihnen zur Verfügung steht?}{sehr schlecht}{ziemlich schlecht}{weder gut noch schlecht}{ziemlich gut}{sehr gut}{kann ich nicht sagen}

\separate

\begin{choicequestion}[6]{Zu wie vielen Ihrer Kommiliton/innen haben Sie ein freundschaftliches Verhältnis?}
	\choiceitemtext{1cm}{1}{Anzahl:}
\end{choicequestion}

\separate

% Skala überdenken!!!
\singlemarkLabelsSixLine{Welchen Stellenwert nimmt Ihr Studium momentan in Ihrem Leben ein?}{unwichtig}{eher unwichtig}{teils teils}{eher wichtig}{sehr wichtig}{kann ich nicht sagen}

\separate

% AUF EINZELFRAGE UMSTELLEN
% Die Frage zielt auf die allgemeine Wahrnehmung der besuchten Schulen!
\begin{markgroupOne}{Wie würden Sie die PH Karlsruhe alles in allem tendenziell beschreiben?}
	\marklineRankingOne{modern}{altmodisch}
	\marklineRankingOne{neu}{abgenutzt}
	\marklineRankingOne{schön}{hässlich}
	\marklineRankingOne{offen}{geschlossen}
	\marklineRankingOne{groß}{klein}
	\marklineRankingOne{einladend}{abweisend}
	\marklineRankingOne{sauber}{schmutzig}
	\marklineRankingOne{monoton}{vielfältig}
\end{markgroupOne}%
