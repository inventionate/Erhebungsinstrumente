% !TEX root = ../Fragebogen_1.tex

% ELTERN
\section{\uppercase{Eltern}}
\vspace{.25cm}

\begin{choicegroup}{Familienstand Ihrer Mutter/ Ihres Vaters:}
	\groupaddchoice{Mutter}
  	\groupaddchoice{Vater}

	\choiceline{ledig ohne feste Partnerschaft}
  	\choiceline{ledig mit fester Partnerschaft}
  	\choiceline{verheiratet}
  	\choiceline{Lebensgemeinschaft}
	\choiceline{unbekannt}

\end{choicegroup}

\separate

\begin{choicegroup}{Ist Ihre Mutter/ Ihr Vater:}
	\groupaddchoice{Mutter}
  	\groupaddchoice{Vater}

	\choiceline{vollzeiterwerbstätig}
  	\choiceline{teilzeiterwerbstätig}
  	\choiceline{arbeitslos/von Kurzarbeit betroffen}
  	\choiceline{Rentner/in/Pernsionär/in}
  	\choiceline{nicht erwerbstätig (z.B. Hausfrau, Hausmann)}	  	\choiceline{unbekannt}
	% An andere Stelle einbinden!
	\choiceline{verstorben}
\end{choicegroup}

\separate

\begin{choicegroup}{Was ist der höchste Schulabschluss Ihrer Mutter/ Ihres Vaters?}
	\groupaddchoice{Mutter}
  	\groupaddchoice{Vater}

	\choiceline{Hauptschulabschluss (mindestens 8. Schuljahr)}
  	\choiceline{Realschulabschluss oder andere Mittelere Reife (10. Schuljahr)}
  	\choiceline{Abitur oder andere Hochschulreife (mindestens 12. Schuljahr)}
  	\choiceline{kein Schulabschluss}
  	\choiceline{mir nicht bekannt}
\end{choicegroup}


\separate

\begin{choicegroup}{Welches ist der höchste berufliche Abschluss Ihrer Mutter/ Ihres Vaters?}
	\groupaddchoice{Mutter}
  	\groupaddchoice{Vater}

	\choiceline{Abschluss einer Fach-, Meister-, Technikerschule, Berufs- oder Fachakademie}
  	\choiceline{Abschluss einer Fachhochschule}
  	\choiceline{Abschluss einer Universität/Kunsthochschule/Pädagogische Hochschule}
  	\choiceline{kein Berufsabschluss}
  	\choiceline{mir nicht bekannt}
\end{choicegroup}

\separate

\begin{choicequestion}[1]{Welchen Beruf üben/übten Ihre Eltern aktuell bzw. zuletzt hauptberuflich aus?}
	\choiceitemtext{1cm}{1}{\makebox[1.5cm]{Mutter:}}
	\choiceitemtext{1cm}{1}{\makebox[1.5cm]{Vater:}}
\end{choicequestion}


\separate

\begin{choicegroup}{Wurden Ihre Eltern in Deutschland geboren?}
	\groupaddchoice{Mutter}
  	\groupaddchoice{Vater}
	\choiceline{Ja}
  	\choiceline{Nein}
\end{choicegroup}

\separate

\begin{choicequestion}[8]{Welche Staatsangehörigkeit haben Ihre Eltern?}
	\choicemulticolitem{8}{deutsch\vspace{.25cm}}
%	\choicemulticolitem{8}{andere}
	\multicolumn{8}{l}{
		\vspace{.25cm}\hspace{1cm}Falls Ihre Eltern eine andere Staatsangehörigkeit haben, welche:
	} \\
	\choiceitemtext{1cm}{4}{\makebox[1.25cm]{Mutter:}}
	\choiceitemtext{1cm}{4}{\makebox[1.25cm]{Vater:}}
\end{choicequestion}

%\separate
%
%\question{\bf Welche Staatsangehörigkeit haben Ihre Eltern? }
%
%\begin{linkanswers}
%\nolink{\pbigbox{deutsche Staatsangehörigkeit}}
%\linkwithquestion{\df{andere Staatsangehörigkeit}}{und zwar:}{\newline\linetext{Vater}\vspace{1.1em} \newline \linetext{Mutter}}
%\end{linkanswers}

\separate

\begin{choicegroup}{Wo wohnen Ihre Eltern?
\newline \footnotesize{Bitte kreuzen Sie -- auch wenn Ihre Eltern zusammenleben -- für beide getrennt das Zutreffende an.}}
	\groupaddchoice{Mutter}
  	\groupaddchoice{Vater}

	\choiceline{in einer Mietwohnung}
	\choiceline{zur Untermiete}
	\choiceline{in einer Eigentumswohnung}
	\choiceline{im eigenen Haus}
\end{choicegroup}

\separate

% Anzahl der Boxen festlegen!
\setcounter{markcheckboxcount}{5}
% Anzahl editiere

\singlemarkLabelsSix{Wie viele Bücher besitzen Ihre Eltern?}{1-10}{11-50}{51-100}{101-250}{251-500}{mehr als 500}

\vspace{-.25cm}
\separate

\singlemarkLabelsSixLine{Wie kommen Ihre Eltern mit dem Geld zurecht, das ihnen zur Verfügung steht?}{sehr schlecht}{ziemlich schlecht}{weder gut noch schlecht}{ziemlich gut}{sehr gut}{kann ich\linebreak nicht sagen}

\vspace{-.25cm}
\separate

\singlemarkLabelsSixLine{Wie würden Sie Ihre Beziehung zu Ihrer Mutter im Allgemeinen beschreiben?}{sehr schlecht}{ziemlich schlecht}{weder gut noch schlecht}{ziemlich gut}{sehr gut}{kann ich\linebreak nicht sagen}

\vspace{-.25cm}
\separate

\singlemarkLabelsSixLine{Wie würden Sie Ihre Beziehung zu Ihrem Vater im Allgemeinen beschreiben?}{sehr schlecht}{ziemlich schlecht}{weder gut noch schlecht}{ziemlich gut}{sehr gut}{kann ich\linebreak nicht sagen}
