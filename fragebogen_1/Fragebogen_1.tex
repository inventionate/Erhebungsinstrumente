\documentclass[
  % Babel language, also used to load translations
  german,
  % Use A4 paper size, you can change this to eg. letterpaper if you need
  % the letter format. The normal methods to modify the paper size should
  % be picked up by SDAPS automatically.
  % a4paper, % setting this might break the example scan unfortunately
  % letterpaper
  %
  % If you need it, you can add a custom barcode at the center
  globalid=Wintersemester-15-16,
  %
  % And the following adds a per sheet barcode at the bottom left
  print_questionnaire_id,
  %
  % You can choose between twoside and oneside. twoside is the default, and
  % requires the document to be printed and scanned in duplex mode.
  twoside,
  %
  % QR Codes instead Barcodes
  style=qr,
  % The following options make sense so that we can get a better feel for the
  % final look.
  pagemark,
  stamp]{sdaps}
\usepackage[utf8]{inputenc}
\usepackage{multicol}

%############ FRAGEBOGEN I ZRS ############%

%%% SCHRIFTARTEN %%%
% Überschriften auf Helvetica setzen.
\usepackage[scaled]{helvet}
% KOMAScript Schriftarten setzen.
\setkomafont{choicefont}{\sffamily}
\setkomafont{marklinequestionfont}{\sffamily}
% Schriftart global auf Sans Serif setzen.
\renewcommand{\familydefault}{\sfdefault}

%%% FARBEN %%%
% Farbe der Fragebogenabschnitte ändern
\definecolor{sectionbgcolor}{rgb}{0,0,0}
\definecolor{sectionfgcolor}{rgb}{255,255,255}

%%% LAYOUT ANPASSUNG %%%
% Trennstrich zwischen den einzelnen Fragen
\renewcommand{\separate}[1][0.12]{\vspace{#1cm}\noindent\makebox[\linewidth]{\rule{\textwidth}{0.5pt}}\vspace{#1cm}}


\author{Fabian Mundt}
\title{Zeit-Raum Studium --- Fragebogen Wintersemester 15/16}

\begin{document}
  % Everything you do should be done inside the questionnaire environment.

  % If you don't like the default text at the beginning of each questionnaire
  % you can remove it with the optional [noinfo] parameter for the environment
  \begin{questionnaire}[noinfo]
    % There is a predefined "info" style to hilight some text.


    %%%%%%%%%%%%%%%%%%%%%%%%%%%%%%%%%%%%%%%%%%%%%%%%%%%

    % PART 0 %
    \begin{center}
		\textbf{\textsc{\Huge Zeit-Raum Studium}} \\
		\vspace{.5cm}
		\textbf{\textsc{\LARGE Fragebogen Wintersemester 15/16}}
	\end{center}
	\vspace{.15cm}

	% Autoreninformationen löschen
	\thispagestyle{empty}
	% Seitenbegrenzungen setzen (für die maschinelle Verarbeitung besser)
	\sdapspagemark

    % INTRO (ALLGEMEINE HINWEISE)
    % !TEX root = Fragebogen_2.tex

% Metadaten für den Auswertungsbericht
\addinfo{Datum}{22.02.2016}
\addinfo{Befragungssituation}{Pretest Folgebefragung Sommersemester 2016}

% Introtext
\begin{info}
\vspace{.5cm}

Sehr geehrte Studentinnen und Studenten,

\vspace{.5em}

der folgende Fragebogen bezieht sich auf Ihren »Zeit-Raum Studium« an der Pädagogischen Hochschule Karlsruhe. Die Erhebung und Auswertung findet im Rahmen meiner Doktorarbeit an der Pädagogischen Hochschule Karlsruhe in der Allgemeinen und Historischen Erziehungswissenschaft statt. Geplant ist, \textbf{Sie in den ersten drei Semestern Ihres Bachelorstudiums forschend zu begleiten}. Dabei interessieren mich insbesondere Ihr biografischer Hintergrund, Ihre alltäglichen Abläufe und Ihre Studienerfahrungen. Ein tieferes Verständnis Ihres Studienalltags soll letztlich Ihnen zugutekommen, indem es ganz konkret auf die Verbesserung der Studienbedingungen vor Ort abzielt.

\vspace{.5em}

Der Fragebogen bezieht sich auf Ihren \textbf{Studienalltag}. Zunächst wird nach Ihrer \textbf{persönlichen Situation} gefragt, danach stehen Fragen zu Ihrer \textbf{gewöhnlichen Studienwoche} im Mittelpunkt. Den Abschluss bilden Fragen zu Ihren \textbf{alltäglichen Studienaktivitäten}.

\vspace{.5em}

Ich bitte Sie, sich Zeit zu nehmen und den Fragebogen ehrlich und vollständig zu bearbeiten. \textbf{Die Auswertung erfolgt anonym.} Da die Studie längsschnittlich (über drei Semester) angelegt ist, wird nach einem \textbf{individuellen Code} gefragt. Dieser dient dazu, die Fragebogen der einzelnen Semester miteinander vergleichen zu können.

\vspace{.5em}

Vielen Dank im Voraus für Ihre Bereitschaft und Unterstützung!

\vspace{-1cm}

\includegraphics[width=0.25\textwidth]{unterschrift.png}

\vspace{-1.5cm}
\end{info}

\vspace{.75cm}

% Bearbeitungshinweise
\begin{info}
\vspace{.5em}

\textbf{\Large Bearbeitungshinweise}

Der Fragebogen wird maschinell erfasst. Um dabei optimale Ergebnisse zu erzielen, verwenden Sie zum Ausfüllen bitte einen \textbf{Kugelschreiber} oder einen \textbf{nicht zu dicken Filzstift} (es muss ein Kreuz erkennbar sein).

\vspace{.5em}

Wichtig ist, dass Sie ein \textbf{kräftiges (möglichst dunkles)} und \textbf{deutlich erkennbares Kreuz} setzen. Bitte verwenden Sie daher keine blassen Stifte (\textbf{keine Bleistifte}).

\vspace{.5em}

Beispiel für ein gut erfassbares Kreuz: {\LARGE \checkedbox{}}

\vspace{.5em}

Sie können einmalig Ihre Markierung korrigieren, indem Sie das Feld komplett ausfüllen.

\vspace{.5em}

Beispiel für eine Korrektur (Kreuz wird nicht erfasst): {\LARGE \correctedbox{}}

\vspace{.5em}

Sollten Sie Ihre Angaben mehrfach korrigieren wollen, notieren Sie ihre Angaben bitte handschriftlich.

\vspace{.5em}

\textbf{Bitte beachten Sie, dass der Fragebogen doppelseitig bedruckt ist. Bitte achten Sie auch darauf, die QR-Codes und die rechtwinkligen Seitenbegrenzungen nicht zu beschädigen.}

\end{info}


    \clearpage
    %%%%%%%%%%%%%%%%%%%%%%%%%%%%%%%%%%%%%%%%%%%%%%%%%%%
	\pagestyle{scrheadings}

    % PART I %
    \begin{center}
		\textbf{\textsc{\Huge Biografischer Hintergrund}}
	\end{center}
	\vspace{.15cm}

    % EIGENE PERSON
    % !TEX root = Fragebogen_1.tex

% EIGENE PERSON
\section{\uppercase{Persönliche Angaben}}
\vspace{-.15cm}

\begin{choicequestion}[6]{Wann wurden Sie geboren?}
	\choiceitemtext{1cm}{1}{Geburtsmonat:}
	\choiceitemtext{1cm}{2}{Geburtsjahr:}
\end{choicequestion}

\vspace{-.15cm}
\separate
\vspace{-.15cm}

\begin{choicequestion}[1]{Welches Geschlecht haben Sie?}
	\choiceitem{männlich}
	\choiceitem{weiblich}
\end{choicequestion}

\vspace{-.15cm}
\separate
\vspace{-.15cm}

\begin{choicequestion}[4]{Welche Staatsangehörigkeit haben Sie?}
	\choicemulticolitem{4}{deutsch} \\
	\choiceitemtext{1cm}{3}{andere:}
\end{choicequestion}

\vspace{-.15cm}
\separate
\vspace{-.15cm}

\begin{choicequestion}[4]{Wo wurden Sie geboren?}
	\choicemulticolitem{4}{Deutschland} \\
	\choiceitemtext{1cm}{3}{anderes Land:}
\end{choicequestion}

\vspace{-.15cm}
\separate
\vspace{-.15cm}

\begin{choicequestion}[1]{Was ist Ihr Familienstand?}
	\choiceitem{ledig ohne feste Partnerschaft}
	\choiceitem{ledig mit fester Partnerschaft}
	\choiceitem{verheiratet}
	\choiceitem{Lebensgemeinschaft}
	\choiceitem{sonstiger}
	\choiceitem{keine Angaben}
\end{choicequestion}

\vspace{-.15cm}
\separate
\vspace{-.15cm}

\begin{choicequestion}[1]{Welcher Religion gehören Sie an?}
	\choiceitem{Christentum}
	\choiceitem{Islam}
	\choiceitem{Judentum}
	\choiceitem{Hinduismus}
	\choiceitem{Buddhismus}
	\choiceitem{andere}
	\choiceitem{keine Religionsgemeinschaft}
	\choiceitem{keine Antwort}
\end{choicequestion}

\vspace{-.15cm}
\separate
\vspace{-.15cm}

\begin{choicequestion}[8]{Wie viele Geschwister haben Sie?}
	\choicemulticolitem{8}{keine} \\
	\choiceitemtext{1cm}{2}{\makebox[3.5cm]{ältere Schwestern:}}
	\choiceitemtext{1cm}{2}{\makebox[3.5cm]{ältere Brüder:}} \\
	\choiceitemtext{1cm}{2}{\makebox[3.5cm]{jüngere Schwestern:}}
	\choiceitemtext{1cm}{2}{\makebox[3.5cm]{jüngere Brüder:}}
\end{choicequestion}

\separate

\begin{choicequestion}[6]{Wie viele eigene Kinder haben Sie?}
	\choicemulticolitem{6}{keine} \\
	\choiceitemtext{1cm}{1}{Falls Sie Kinder haben, wie viele:}
\end{choicequestion}


    % ELTERN
    % !TEX root = Fragebogen_1.tex

% ELTERN
\section{\uppercase{Eltern}}
\vspace{.25cm}


\begin{choicegroup}{Ist Ihre Mutter/ Ihr Vater verstorben?}
	\groupaddchoice{Mutter}
	\groupaddchoice{Vater}

	\choiceline{ja}
	\choiceline{nein}
	\choiceline{unbekannt}
\end{choicegroup}

\separate

\begin{choicegroup}{Was ist der Familienstand Ihrer Mutter/ Ihres Vaters?}
	\groupaddchoice{Mutter}
	\groupaddchoice{Vater}

	\choiceline{verheiratet mit Ihrem biologischem Elternteil}
	\choiceline{verheiratet mit neuem Partner}
	\choiceline{feste Partnerschaft mit Ihrem biologischen Elternteil}
	\choiceline{feste Partnerschaft mit neuem Partner}
	\choiceline{ledig ohne feste Partnerschaft}
	\choiceline{unbekannt}
\end{choicegroup}

\separate

\begin{choicegroup}{Wo wohnen Ihre Eltern?
\newline \footnotesize{Bitte kreuzen Sie -- auch wenn Ihre Eltern zusammenleben -- für beide getrennt das Zutreffende an.}}
	\groupaddchoice{Mutter}
  	\groupaddchoice{Vater}

	\choiceline{in einer Mietwohnung}
	\choiceline{zur Untermiete}
	\choiceline{in einer Eigentumswohnung}
	\choiceline{im eigenen Haus}
\end{choicegroup}

\separate

\begin{choicegroup}{Was ist der höchste Schulabschluss Ihrer Mutter/ Ihres Vaters?}
	\groupaddchoice{Mutter}
  	\groupaddchoice{Vater}

	\choiceline{kein Schulabschluss}
	\choiceline{Hauptschulabschluss (mindestens 8. Schuljahr)}
  	\choiceline{Realschulabschluss oder andere Mittlere Reife (10. Schuljahr)}
  	\choiceline{Abitur oder andere Hochschulreife (mindestens 12. Schuljahr)}
  	\choiceline{mir nicht bekannt}
\end{choicegroup}

\vspace{-.25cm}
\separate
\vspace{-.25cm}

\begin{choicegroup}{Welches ist der höchste nachschulische Abschluss Ihrer Mutter/ Ihres Vaters?}
	\groupaddchoice{Mutter}
  	\groupaddchoice{Vater}

	\choiceline{kein Abschluss}
	\choiceline{Abschluss einer Fach-, Meister-, Technikerschule, Berufs- oder Fachakademie}
  	\choiceline{Abschluss einer Fachhochschule}
  	\choiceline{Abschluss einer Universität/Kunsthochschule/Pädagogische Hochschule}
  	\choiceline{mir nicht bekannt}
\end{choicegroup}

\vspace{-.25cm}
\separate
\vspace{-.25cm}

\begin{choicegroup}{Wie erwerben Ihre Mutter/ Ihr Vater ihren Lebensunterhalt?}
	\groupaddchoice{Mutter}
  	\groupaddchoice{Vater}

	\choiceline{vollzeiterwerbstätig}
  	\choiceline{teilzeiterwerbstätig}
  	\choiceline{arbeitslos/von Kurzarbeit betroffen}
  	\choiceline{Rentner/in/Pensionär/in}
  	\choiceline{nicht erwerbstätig (z.B. Hausfrau, Hausmann)}	  	\choiceline{unbekannt}
	% An andere Stelle einbinden!
	% \choiceline{verstorben}
\end{choicegroup}

\separate

\begin{choicequestion}[1]{Welchen Beruf üben/übten Ihre Eltern aktuell bzw. zuletzt hauptberuflich aus?}
	\choiceitemtext{1cm}{1}{\makebox[1.5cm]{Mutter:}}
	\choiceitemtext{1cm}{1}{\makebox[1.5cm]{Vater:}}
\end{choicequestion}

\separate

\begin{choicegrouptext}{Welche Staatsangehörigkeit haben Ihre Eltern?}
	\groupaddchoice{deutsch}
  	\groupaddtextchoice{andere}

	\choiceline{Mutter}
  	\choiceline{Vater}
\end{choicegrouptext}

\separate

\begin{choicegroup}{Wurden Ihre Eltern in Deutschland geboren?}
	\groupaddchoice{Ja}
  	\groupaddchoice{Nein}
	\choiceline{Mutter}
  	\choiceline{Vater}
\end{choicegroup}

\separate

% Anzahl der Boxen festlegen!
\setcounter{markcheckboxcount}{5}
% Anzahl editiere

\singlemarkLabelsSix{Wie viele Bücher besitzen Ihre Eltern?}{1-10}{11-50}{51-100}{101-250}{251-500}{mehr als 500}

\separate

\singlemarkLabelsSixLine{Was würden Sie sagen, wie Ihre Eltern alles in allem mit dem Geld zurechtkommen, das ihnen zur Verfügung steht?}{sehr schlecht}{ziemlich schlecht}{weder gut noch schlecht}{ziemlich gut}{sehr gut}{kann ich\linebreak nicht sagen}

\separate

\singlemarkLabelsSixLine{Wie würden Sie Ihre Beziehung zu Ihrer Mutter im Allgemeinen beschreiben?}{sehr schlecht}{ziemlich schlecht}{weder gut noch schlecht}{ziemlich gut}{sehr gut}{kann ich\linebreak nicht sagen}

\separate

\singlemarkLabelsSixLine{Wie würden Sie Ihre Beziehung zu Ihrem Vater im Allgemeinen beschreiben?}{sehr schlecht}{ziemlich schlecht}{weder gut noch schlecht}{ziemlich gut}{sehr gut}{kann ich\linebreak nicht sagen}


	% KIDNHEIT UND JUGEND
	\input{biografischer_hintergrund_kindheit}

	% SCHULZEIT
	% !TEX root = Fragebogen_1.tex

\section{\uppercase{Schulzeit}}
\vspace{.25cm}

\begin{choicequestion}[1]{Welche Schulform(en) haben Sie als Schüler/in besucht?\newline\footnotesize{Mehrfachnennungen sind möglich.}}
	\choiceitem{Hauptschule/Werkrealschule oder vergleichbar}
	\choiceitem{Realschule/Mittelschule oder vergleichbar}
	\choiceitem{Berufliches Gymnasium}
	\choiceitem{Gymnasium}
	\choiceitem{Gesamtschule}
	\choiceitem{Abendschule}
	\choiceitem{Privatschule}
	% Eventuell rechten Abstand verändern!
	\choiceitemtext{1cm}{1}{andere:}
\end{choicequestion}

\vspace{-.15cm}
\separate
\vspace{-.15cm}

\begin{choicegroup}{Wie weit entfernt lag Ihre Schule von Ihrem Zuhause?
	\newline\footnotesize Falls Sie die Schule gewechselt haben, beziehen Sie sich auf die von Ihnen am längsten besuchte Schule.}
	\groupaddchoice{unter 5km}
	\groupaddchoice{5 bis 10km}
	\groupaddchoice{10 bis 15km}
	\groupaddchoice{15 bis 20km}
	\groupaddchoice{mehr als 20km}

  	\choiceline{Grundschule}
  	\choiceline{Sekundarstufe 1}
  	\choiceline{Sekundarstufe 2}
	% Weiterbildungsangebote hier ausklammern?!
\end{choicegroup}

% Hier auch nach der Anfahrtsdauer und dem Anfahrtsgefährt fragen?
% Bzw. wie man im allgemeinen zur Schule gekommen ist
\vspace{-.15cm}
\separate
\vspace{-.15cm}

\begin{choicegroup}{Wie lange brauchten Sie durchschnittlich, um von Ihrem Zuhause zu Ihrer Schule zu kommen?
	\newline\footnotesize Falls Sie die Schule gewechselt haben, beziehen Sie sich auf die von Ihnen am längsten besuchte Schule.}
	\groupaddchoice{unter 15min}
	\groupaddchoice{15 bis 30min}
	\groupaddchoice{30 bis 45min}
	\groupaddchoice{45 bis 60min}
	\groupaddchoice{mehr als 60min}

	\choiceline{Grundschule}
  	\choiceline{Sekundarstufe 1}
  	\choiceline{Sekundarstufe 2}
	% Weiterbildungsangebote hier ausklammern?!
\end{choicegroup}

\vspace{-.15cm}
\separate
\vspace{-.15cm}

\begin{choicegroup}{Mit welchem Fortbewegungsmittel sind Sie hauptsächlich zur Schule gekommen?
\newline\footnotesize{Mehrfachnennungen sind möglich.}}
	\groupaddchoice{Auto} % Benennen, dass es das Mitfahren ist?
	\groupaddchoice{Motorrad/Roller}
	\groupaddchoice{Bus}
	\groupaddchoice{Bahn}
	\groupaddchoice{Fahrrad}
	\groupaddchoice{zu Fuß}

 	\choiceline{Grundschule}
  	\choiceline{Sekundarstufe 1}
  	\choiceline{Sekundarstufe 2}
	% Weiterbildungsangebote hier ausklammern?!
\end{choicegroup}

\vspace{-.15cm}
\separate
\vspace{-.15cm}

\begin{choicequestion}[1]{Haben Sie an einem Schüleraustausch teilgenommen?}
	\choiceitem{Ja}
	\choiceitem{Nein}
\end{choicequestion}

\vspace{-.15cm}
\separate
\vspace{-.15cm}

\begin{choicequestion}[1]{Wurden an Ihrer Schule alternative Zeitmodelle umgesetzt (keine »einstündige« 45, 50 oder 55 Minuten Taktung)?}
	\choiceitem{Ja}
	\choiceitem{Nein}
\end{choicequestion}

\vspace{-.15cm}
\separate
\vspace{-.15cm}

% Die Frage zielt auf die allgemeine Wahrnehmung der besuchten Schulen!
\begin{markgroupThree}{Wie würden Sie die von Ihnen besuchten Schulen alles in allem tendenziell beschreiben?
	\newline\footnotesize Falls Sie die Schule gewechselt haben, beziehen Sie sich auf die von Ihnen am längsten besuchte Schule.}{Grundschule}{Sekundarstufe 1}{Sekundarstufe 2}
	\marklineRanking{modern}{altmodisch}
	\marklineRanking{neu}{abgenutzt}
	\marklineRanking{schön}{hässlich}
	\marklineRanking{offen}{geschlossen}
	\marklineRanking{groß}{klein}
	\marklineRanking{einladend}{abweisend}
	\marklineRanking{sauber}{schmutzig}
	\marklineRanking{monoton}{vielfältig}
\end{markgroupThree}%

\vspace{-.15cm}
\separate
\vspace{-.25cm}

\begin{choicequestion}[1]{Welche Art der Hochschulreife besitzen Sie?}
	\choiceitem{allgemeine Hochschulreife}
	\choiceitem{fachgebundene Hochschulreife}
	\choiceitem{Fachhochschulreife}
	\choiceitem{andere Studienberechtigung}
\end{choicequestion}

\vspace{-.15cm}
\separate
\vspace{-.25cm}

% ÜBERDENKEN OB SINNVOLL (mit Blick auf die Identifikation)
\begin{choicequestion}[3]{Welche Durchschnittsnote hatten Sie in dem Abschlusszeugnis, das Sie zur Aufnahme eines Studiums berechtigt?
\newline\footnotesize{Tragen Sie bitte die Note (z.B. 2,5) ein.}}
	\choiceitemtext{1cm}{1}{Note:}
\end{choicequestion}

\vspace{-.15cm}
\separate
\vspace{-.25cm}

\begin{choicequestion}[3]{Wann haben Sie Ihre Hochschulzugangsberechtigung erworben?}
	\choiceitemtext{1cm}{1}{Jahr:}
\end{choicequestion}

\vspace{-.15cm}
\separate
\vspace{-.25cm}

\begin{choicequestion}[1]{In welchem Bundesland/Land haben Sie die Berechtigung zum Hochschulstudium erworben?}
	\choiceitem{Baden-Württemberg}
	\choiceitemtext{1cm}{1}{anderes:}
\end{choicequestion}

\vspace{-.15cm}
\separate
\vspace{-.25cm}

\singlemarkLabelsSixLine{Wie würden Sie Ihre Beziehung zu Ihrer Schulzeit im Allgemeinen beschreiben?}{sehr schlecht}{ziemlich schlecht}{weder gut noch schlecht}{ziemlich gut}{sehr gut}{kann ich nicht sagen}

\vspace{-.15cm}
\separate
\vspace{-.25cm}

\singlemarkLabelsSixLine{Wie fühlen Sie sich durch Ihre Schulzeit auf das Studium vorbereitet?}{sehr schlecht}{ziemlich schlecht}{weder gut noch schlecht}{ziemlich gut}{sehr gut}{kann ich nicht sagen}


    %%%%%%%%%%%%%%%%%%%%%%%%%%%%%%%%%%%%%%%%%%%%%%%%%%%

	\newpage
	% PART II %
    \begin{center}
		\textbf{\textsc{\Huge Übergang Schule – Hochschule}}
	\end{center}
	\vspace{.15cm}

	% ÜBERGANG SCHULE – HOCHSCHULE
	\input{uebergang_schule_hochschule_uebergang}

    % ÜBERGANG SCHULE – HOCHSCHULE
	% !TEX root = Fragebogen_1.tex

\vspace{-.25cm}
\section{\uppercase{Studienwahlmotive}}
\vspace{-.25cm}

\begin{choicegroupReasons}{Wie wichtig waren die folgenden Gründe für Ihre Entscheidung, an der PH Karlsruhe zu studieren?}
\label{studienort}
	\groupaddchoiceReason{unwichtig}
	\groupaddchoiceReason{eher\linebreak unwichtig}
	\groupaddchoiceReason{teils, teils}
	\groupaddchoiceReason{eher wichtig}
	\groupaddchoiceReason{sehr wichtig}
	\groupaddchoiceReason{kann ich nicht sagen}

	\choicelineReason{guter Ruf der PH Karlsruhe}
  	\choicelineReason{vielfältiges Lehrangebot}
  	\choicelineReason{gute Ausstattung der PH}
  	\choicelineReason{Freizeitangebot der PH}
  	\choicelineReason{Atmosphäre in Karlsruhe (student. Leben usw.)}
  	\choicelineReason{Nähe zum Heimatort}
  	\choicelineReason{günstige Lebensbedingungen in Karlsruhe}
  	\choicelineReason{Partner/Partnerin lebt am Hochschulort}
  	\choicelineReason{Eltern/Verwandte/Freunde leben am Hochschulort}
  	\choicelineReason{PH bzw. Karlsruhe ist mir vertraut}
  	\choicelineReason{Informationen der Studienberatung}
\end{choicegroupReasons}

\vspace{-.15cm}
\separate
\vspace{-.15cm}

\begin{choicequestion}[2]{Welcher der oben genannten Gründe war für Ihre Hochschulwahl der wichtigste?}
	\choiceitemtext{1cm}{1}{Bitte vorangestellten Buchstaben aus Frage \ref{studienort} eintragen:}
\end{choicequestion}

\vspace{-.15cm}
\separate
\vspace{-.15cm}

% UMFORDMULIEREN, da der einleitende Text eher gestrichen werden soll!!!
\begin{choicegroupReasons}{Wie wichtig sind die folgenden Gründe für die Wahl Ihres Studiums?}
\label{studienwahl}
	\groupaddchoiceReason{unwichtig}
	\groupaddchoiceReason{eher\linebreak unwichtig}
	\groupaddchoiceReason{teils, teils}
	\groupaddchoiceReason{eher wichtig}
	\groupaddchoiceReason{sehr wichtig}
	\groupaddchoiceReason{kann ich nicht sagen}

	\choicelineReason{fachspezifisches Interesse}
	\choicelineReason{das Studium entspricht meinen Begabungen}
	\choicelineReason{zu sozialen Veränderungen beitragen}
	\choicelineReason{anderen helfen}
	\choicelineReason{Verwandte/Bekannte arbeiten in entspr. Berufen}
	\choicelineReason{wissenschaftliches Interesse}
	\choicelineReason{angesehener Beruf bekommen}
	\choicelineReason{gesicherte Berufsposition erhalten}
	\choicelineReason{gute Verdienstchancen erreichen}
	\choicelineReason{bestimmter Berufswunsch}
	\choicelineReason{mir erscheint mein Studium als das kleinste Übel}
	\choicelineReason{viel Umgang mit Menschen haben}
	\choicelineReason{günstige Chancen auf dem Arbeitsmarkt}
\end{choicegroupReasons}

\vspace{-.15cm}
\separate
\vspace{-.15cm}

\begin{choicequestion}[2]{Welcher der oben genannten Gründe ist für Ihre Studiengangswahl der wichtigste?}
	\choiceitemtext{1cm}{1}{Bitte vorangestellten Buchstaben aus Frage \ref{studienwahl} eintragen:}
\end{choicequestion}


    % ÜBERGANG SCHULE – HOCHSCHULE
	% !TEX root = Fragebogen_1.tex

\section{\uppercase{Studienwahl}}
\vspace{.25cm}

\begin{choicequestion}[1]{Welchen Bachelor of Education studieren Sie?}
	\choiceitem{Primarstufe}
	\choiceitem{Sekundarstufe}
\end{choicequestion}

\separate

\begin{choicequestion}[1]{Studieren Sie mit der Profilierung Europalehramt?}
	\choiceitem{ja}
	\choiceitem{nein}
\end{choicequestion}

\separate

\begin{choicequestion}[1]{Planen Sie als Lehrer/in tätig zu sein (den Master of Education anzuschließen)?}
	\choiceitem{ja}
	\choiceitem{nein}
\end{choicequestion}

\separate

%todo Klären, wie mit dem Wahlbereich Primarstufe MATHE/DEUTSCH umzugehen ist.
\begin{choicequestion}[1]{Welche Fächerkombination haben Sie gewählt?}
	\choiceitemtext{1cm}{1}{1. Fach:}
	\choiceitemtext{1cm}{1}{2. Fach:}
\end{choicequestion}

\separate

\begin{choicequestion}[1]{Ist das die von Ihnen am meisten gewünschte Fächerkombination?}
	\choiceitem{ja}
	\choiceitem{nein}
\end{choicequestion}

\separate

\begin{choicequestion}[1]{Handelt es sich um Ihr erstes Studiensemester an der PH Karlsruhe?}
	\choiceitem{ja}
	\choiceitem{nein}
\end{choicequestion}

\separate

\begin{choicequestion}[1]{Haben Sie an der Orientierungsphase (O-Phase) der PH/StuVe teilgenommen?}
	\choiceitem{vollständig (Montag bis Samstag)}% TAGE KLÄREN
	\choiceitem{teilweise (mindestens ein Tag)}
	\choiceitem{gar nicht}
\end{choicequestion}


    %%%%%%%%%%%%%%%%%%%%%%%%%%%%%%%%%%%%%%%%%%%%%%%%%%%

	\newpage
	% PART III %
    \begin{center}
		\textbf{\textsc{\Huge Studienalltag}}
	\end{center}
    \vspace{.15cm}

	% PERSÖNLICHE SITUATION
	% !TEX root = Fragebogen_2.tex

% Persönliche Situation
\section{Persönliche Situation}
\vspace{.25cm}

\begin{choicequestion}[1]{Wo wohnen Sie \underline{während der Vorlesungszeit?}}
	\choiceitem{bei meinen Eltern}
	\choiceitem{in einer WG}
	\choiceitem{in einem Wohnheim}
	\choiceitem{allein in einer eigenen Wohnung}
	\choiceitem{mit meinem Partner/meiner Partnerin}
	\choiceitem{woanders}
\end{choicequestion}

\separate

\begin{choicequestion}[6]{Wie viele Kilometer sind es ungefähr von Ihrem Wohnort \underline{während der Vorlesungszeit} bis zur Hochschule?}
	\choiceitemtext{1cm}{1}{Kilometer:}
\end{choicequestion}

\separate

\begin{choicequestion}[6]{Wie lange brauchen Sie ungefähr von Ihrem Wohnort \underline{während der Vorlesungszeit} bis zur Hochschule?}
	\choiceitemtext{1cm}{1}{Minuten:~~~}
\end{choicequestion}

\separate

\begin{choicequestion}[1]{Wo wohnen Sie aller Voraussicht nach \underline{während der »Semesterferien«?}}
	\choiceitem{bei meinen Eltern}
	\choiceitem{in einer WG}
	\choiceitem{in einem Wohnheim}
	\choiceitem{allein in einer eigenen Wohnung}
	\choiceitem{mit meinem Partner/meiner Partnerin}
	\choiceitem{woanders}
	%\choiceitemtext{1cm}{1}{anderer Ort:}
\end{choicequestion}

\separate

\begin{choicequestion}[6]{Wie viele Kilometer sind es ungefähr von Ihrem Wohnort \underline{während der »Semesterferien«} bis zur Hochschule?}
	\choiceitemtext{1cm}{1}{Kilometer:}
\end{choicequestion}

\separate

\begin{choicequestion}[6]{Wie lange brauchen Sie ungefähr von Ihrem Wohnort \underline{während der »Semesterferien«} bis zur Hochschule?}
	\choiceitemtext{1cm}{1}{Minuten:~~~}
\end{choicequestion}

\separate

\begin{choicegroup}{Wie finanzieren Sie zur Zeit Ihr Studium?}
	\groupaddchoice{nein, dadurch nicht}
	\groupaddchoice{ja, teilweise}
	\groupaddchoice{ja, hauptsächlich}

	\choiceline{durch Unterstützung der Eltern}
	\choiceline{durch Einkommen des/ der (Ehe-)Partners/-in} %/der (Ehe-)Partnerin}
	\choiceline{durch BAföG}
	\choiceline{durch Studienkredit/ Bildungskredit}
	\choiceline{durch Stipendien (Förderwerke, Stiftungen,…)}
%	\choiceline{durch die Arbeit als studentische Hilfskraft}
	\choiceline{durch eigene Arbeit während der Vorlesungszeit}
	\choiceline{durch eigene Arbeit während der Semesterferien}
	\choiceline{durch anderes}
\end{choicegroup}

\vspace{-.15cm}
\separate
\vspace{-.25cm}

\textbox*{1cm}{Falls Sie einer bezahlten Arbeit nachgehen, um welche handelt es sich?}

\vspace{-.1cm}
\separate
\vspace{-.1cm}

\singlemarkLabelsSixLine{Wie kommen Sie insgesamt mit dem Geld zurecht, das Ihnen zur Verfügung steht?}{sehr schlecht}{ziemlich schlecht}{weder gut noch schlecht}{ziemlich gut}{sehr gut}{kann ich nicht sagen}

\vspace{-.35cm}
\separate
\vspace{-.25cm}

\begin{choicequestion}[6]{Wie viele Ihrer Kommiliton/innen würden Sie als Freund/innen bezeichnen?}
	\choiceitemtext{1cm}{1}{Anzahl:}
\end{choicequestion}

\vspace{-.15cm}
\separate
\vspace{-.25cm}

\singlemarkLabelsSixLine{Welchen Stellenwert nimmt Ihr Studium momentan in Ihrem Leben ein?}{unwichtig}{eher unwichtig}{teils/teils}{eher wichtig}{sehr wichtig}{kann ich nicht sagen}

\vspace{-.35cm}
\separate
\vspace{-.35cm}

% NEUE FRAGEN THEMATISIEREN!!!
\singlemarkLabelsSixLine{Wie zufrieden sind Sie alles in allem mit Ihrem Studium?}{sehr unzufrieden}{eher unzufrieden}{teils/teils}{eher zufrieden}{ sehr zufrieden}{kann ich nicht sagen}

\vspace{-.35cm}
\separate
\vspace{-.35cm}

\begin{choicequestion}[1]{Haben Sie Ihre Fächerkombination gewechselt?}
	\choiceitem{ja}
	\choiceitem{ich überlege es mir}
	\choiceitem{nein}
\end{choicequestion}

\vspace{-.15cm}
\separate
\vspace{-.35cm}

\begin{choicequestion}[1]{Spielen Sie mit dem Gedanken Ihr Studium abzubrechen?}
	\choiceitem{ja}
	\choiceitem{manchmal}
	\choiceitem{nein}
\end{choicequestion}

\vspace{-.1cm}


	% EINE GWÖHNLICHE STUDIENWOCHE
	% !TEX root = Fragebogen_2.tex

\pagebreak
% gewöhnliche Studienwoche
\vspace{-.25cm}
\section{Gewöhnliche Studienwoche}

\question{Wie viel Zeit verbringen Sie pro Wochentag mit den angegebenen Tätigkeiten?
\newline\small{Bitte \underline{verteilen Sie die 24 Stunden eines Tages}. Beachten Sie das von Ihren Mentor/innen eingeblendete Beispiel.}}
% MIT SDAPS AUTOMATISIEREN!!!!
% >{\centering\arraybackslash} Array Syntax um Spalten zu formatieren
\begin{tabular}{|m{6.1cm}|*{7}{>{\centering\bf\arraybackslash}p{1.25cm}|}}

\hline
							& Mo 	& Di	& Mi 	& Do 	& Fr 	& Sa 	& So \\
% \hline
% \vspace{.50cm}
% \textbf{Zeit an der Hochschule} %\newline\footnotesize{(Wie lange sind auf dem Hochschulgelände)}
% \vspace{.50cm} 				& 		&		& 		& 		& 		&		& 	 \\
\hline
\vspace{.50cm}
\textbf{Lehrveranstaltungen} %\newline\footnotesize{(Vorlesungen, Seminare etc.)}
\vspace{.50cm} 				& 		&		& 		& 		& 		&		& 	 \\
\hline
\vspace{.50cm}
\textbf{Zeit zwischen den Veranstaltungen}\nolinebreak %\newline\footnotesize{(»Wartezeiten«)}
\vspace{.50cm} 				& 		&		& 		& 		& 		&		& 	 \\
\hline
\vspace{.50cm}
\textbf{Selbststudium} % \newline\footnotesize{(Lernen für das Studium)}
\vspace{.50cm} 				& 		&		& 		& 		& 		&		& 	 \\
\hline
\vspace{.50cm}
\textbf{Fahrtzeit}% \newline\footnotesize{(Zeit zur PH und zurück zur Wohnung.)}
\vspace{.50cm} 				& 		&		& 		& 		& 		&		& 	 \\
\hline
\vspace{.50cm}
\textbf{Arbeitszeit}% \newline\footnotesize{(Zeit zum Geld verdienen.)}
\vspace{.50cm} 				& 		&		& 		& 		& 		&		& 	 \\
\hline
\vspace{.50cm}
\textbf{Freizeit (inkl. Haushalt)} %\newline\footnotesize{(Zeit für Ihre Hobbies etc.)}
\vspace{.50cm} 				& 		&		& 		& 		& 		&		& 	 \\
\hline
\vspace{.50cm}
\textbf{Schlafen}
\vspace{.50cm} 				& 		&		& 		& 		& 		&		& 	 \\
\hline
% \vspace{.50cm}
% \textbf{Zeit im eigenen Wohnbereich} %\newline\footnotesize{~}
% \vspace{.50cm} 				& 		&		& 		& 		& 		&		& 	 \\
% \hline
\end{tabular}
\vspace{.15cm}

\separate
\vspace{-.1cm}

\begin{choicequestion}[6]{An wie vielen Lehrveranstaltungen nehmen Sie dieses Semester teil?\newline\small{Tutorien fügen Sie bitte in Klammer hinzu. Für 9 Lehrveranstaltungen und 2 Tutorien: 9 (2).}}
	\choiceitemtext{1cm}{1}{Anzahl:}
\end{choicequestion}

\vspace{-.1cm}
\separate
\vspace{-.1cm}

\begin{choicequestion}[1]{Sind Sie alles in allem mit Ihrem Lehrveranstaltungs\underline{plan} zufrieden?}
	\choiceitem{ja}
	\choiceitem{teilweise}
	\choiceitem{nein}
\end{choicequestion}

\vspace{-.1cm}
\separate
\vspace{-.1cm}

\begin{choicequestion}[1]{Mit welchem Fortbewegungsmittel kommen Sie hauptsächlich zur Hochschule?}
	\choiceitem{Auto}
	\choiceitem{Motorrad/Roller}
	\choiceitem{Bus}
	\choiceitem{Bahn}
	\choiceitem{Fahrrad}
	\choiceitem{zu Fuß}
\end{choicequestion}

\vspace{-.1cm}


	% GEWÖHNLICHE STUDIENTÄTIGKEITEN INNERHALB EINER WOCHE
	\input{studienalltag_studientaetigkeiten}

	% GEWÖHNLICHE FREIZEITAKTIVITÄTEN INNERHALB EINER WOCHE
	\input{studienalltag_freizeitaktivitaeten}

    %%%%%%%%%%%%%%%%%%%%%%%%%%%%%%%%%%%%%%%%%%%%%%%%%%%

    \newpage
    % PART IV %
    \begin{center}
		\textbf{\textsc{\Huge Schluss}}
	\end{center}
	\vspace{.15cm}

	% EXTRO (IDENTIFIKATION UND ANONYMISIERUNGSHINWEISE)
	% !TEX root = Fragebogen_2.tex

% Extro
\vspace{-.25cm}
\section{Weitere Teilnahme an der Studie}
\vspace{.25cm}

\begin{choicequestion}[4]{Die Studie versucht, Sie im Rahmen der ersten drei Studiensemester zu begleiten. Um die einzelnen Befragungen zuordnen zu können, wird ein anonymisierter Zuordnungsschlüssel benötigt.}
	\choiceitemtext{1cm}{1}{\makebox[11cm]{Erste \underline{zwei Buchstaben} des Vornamens Ihrer \underline{Mutter} (z.B. \textit{JU} für \textit{\underline{Ju}lia}):\hfill}}\\
	\choiceitemtext{1cm}{1}{\makebox[11cm]{Erste \underline{zwei Buchstaben} des Vornamens Ihres \underline{Vaters} (z.B. \textit{PA} für \textit{\underline{Pa}ul}):\hfill}}\\
	\choiceitemtext{1cm}{1}{\hspace*{-0.2cm}\makebox[11.2cm]{Der \underline{Tag} Ihres Geburtsdatums (z.B. \textit{17} für den \textit{\underline{17}.09.1997}):\hfill}}
\end{choicequestion}

\separate

\vspace{.25cm}
\section{Rückmeldung}
\vspace{.25cm}


\textbox*{8.5cm}{Wenn Sie noch Ergänzungen, Anmerkungen oder kritische Hinweise haben, bitte ich Sie, mir diese hier mitzuteilen.}

\separate

\textbf{Vielen Dank, dass Sie mich durch das Ausfüllen des Fragebogens bei meiner Doktorarbeit unterstützt haben!}

\vspace{-1.5cm}
\flushright
\includegraphics[width=0.25\textwidth]{unterschrift.png}
\vspace{-1.5cm}
\flushright
\href{mailto:fabian.mundt@ph-karlsruhe.de}{\nolinkurl{fabian.mundt@ph-karlsruhe.de}}


    %%%%%%%%%%%%%%%%%%%%%%%%%%%%%%%%%%%%%%%%%%%%%%%%%%%

  \end{questionnaire}
\end{document}
