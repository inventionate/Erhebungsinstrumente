% !TEX root = Fragebogen_2.tex

% Metadaten für den Auswertungsbericht
\addinfo{Datum}{22.02.2016}
\addinfo{Befragungssituation}{Pretest Folgebefragung Sommersemester 2016}

% Introtext
\begin{info}
\vspace{.5cm}

Sehr geehrte Studentinnen und Studenten,

\vspace{.5em}

der folgende Fragebogen bezieht sich auf Ihren »Zeit-Raum Studium« an der Pädagogischen Hochschule Karlsruhe. Die Erhebung und Auswertung findet im Rahmen meiner Doktorarbeit an der Pädagogischen Hochschule Karlsruhe in der Allgemeinen und Historischen Erziehungswissenschaft statt. Geplant ist, \textbf{Sie in den ersten drei Semestern Ihres Bachelorstudiums forschend zu begleiten}. Dabei interessieren mich insbesondere Ihr biografischer Hintergrund, Ihre alltäglichen Abläufe und Ihre Studienerfahrungen. Ein tieferes Verständnis Ihres Studienalltags soll letztlich Ihnen zugutekommen, indem es ganz konkret auf die Verbesserung der Studienbedingungen vor Ort abzielt.

\vspace{.5em}

Der Fragebogen bezieht sich auf Ihren \textbf{Studienalltag}. Zunächst wird nach Ihrer \textbf{persönlichen Situation} gefragt, danach stehen Fragen zu Ihrer \textbf{gewöhnlichen Studienwoche} im Mittelpunkt. Den Abschluss bilden Fragen zu Ihren \textbf{alltäglichen Studienaktivitäten}.

\vspace{.5em}

Ich bitte Sie, sich Zeit zu nehmen und den Fragebogen ehrlich und vollständig zu bearbeiten. \textbf{Die Auswertung erfolgt anonym.} Da die Studie längsschnittlich (über drei Semester) angelegt ist, wird nach einem \textbf{individuellen Code} gefragt. Dieser dient dazu, die Fragebogen der einzelnen Semester miteinander vergleichen zu können.

\vspace{.5em}

Vielen Dank im Voraus für Ihre Bereitschaft und Unterstützung!

\vspace{-1cm}

\includegraphics[width=0.25\textwidth]{unterschrift.png}

\vspace{-1.5cm}
\end{info}

\vspace{.75cm}

% Bearbeitungshinweise
\begin{info}
\vspace{.5em}

\textbf{\Large Bearbeitungshinweise}

Der Fragebogen wird maschinell erfasst. Um dabei optimale Ergebnisse zu erzielen, verwenden Sie zum Ausfüllen bitte einen \textbf{Kugelschreiber} oder einen \textbf{nicht zu dicken Filzstift} (es muss ein Kreuz erkennbar sein).

\vspace{.5em}

Wichtig ist, dass Sie ein \textbf{kräftiges (möglichst dunkles)} und \textbf{deutlich erkennbares Kreuz} setzen. Bitte verwenden Sie daher keine blassen Stifte (\textbf{keine Bleistifte}).

\vspace{.5em}

Beispiel für ein gut erfassbares Kreuz: {\LARGE \checkedbox{}}

\vspace{.5em}

Sie können einmalig Ihre Markierung korrigieren, indem Sie das Feld komplett ausfüllen.

\vspace{.5em}

Beispiel für eine Korrektur (Kreuz wird nicht erfasst): {\LARGE \correctedbox{}}

\vspace{.5em}

Sollten Sie Ihre Angaben mehrfach korrigieren wollen, notieren Sie ihre Angaben bitte handschriftlich.

\vspace{.5em}

\textbf{Bitte beachten Sie, dass der Fragebogen doppelseitig bedruckt ist. Bitte achten Sie auch darauf, die QR-Codes und die rechtwinkligen Seitenbegrenzungen nicht zu beschädigen.}

\end{info}
